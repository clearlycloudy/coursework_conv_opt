\documentclass[12pt,letter]{article}

%% \usepackage[fleqn]{amsmath}
\usepackage[margin=1in]{geometry}
\usepackage{amsmath,amsfonts,amsthm,bm}
\usepackage{breqn}
\usepackage{amsmath}
\usepackage{amssymb}
\usepackage{tikz}
\usepackage{algorithm2e}
\usepackage{siunitx}
\usepackage{graphicx}
\usepackage{subcaption}
%% \usepackage{datetime}
\usepackage{multirow}
\usepackage{multicol}
\usepackage{mathrsfs}
\usepackage{fancyhdr}
\usepackage{fancyvrb}
\usepackage{parskip} %turns off paragraph indent
\usepackage{float}

\pagestyle{fancy}

\usetikzlibrary{arrows}

\DeclareMathOperator*{\argmin}{argmin}
\newcommand*{\argminl}{\argmin\limits}

\newcommand{\mathleft}{\@fleqntrue\@mathmargin0pt}
\newcommand{\R}{\mathbb{R}}
\newcommand{\Z}{\mathbb{Z}}
\newcommand{\N}{\mathbb{N}}
\newcommand{\ppartial}[2]{\frac{\partial #1}{\partial #2}}
\newcommand{\norm}[1]{\|#1\|}
\newcommand{\set}[1]{\{#1\}}
\newcommand{\notimplies}{\;\not\!\!\!\implies}

\setcounter{MaxMatrixCols}{20}

\begin {document}

  % \begin{cases}
  %     0, & \text{if}\ a=1 \\
  %     1, & \text{otherwise}
  %   \end{cases}

\lhead{Convex Optimization - HW2}
\rhead{(Bill) Yuan Liu, 996954078, 2020/02/22}

\begin{enumerate}
  
\item Q 4.11(a,c,e) textbook
  
\item Q 4.16 textbook
  
\item Q 4.21(a) textbook
  
\item Q 4.25 textbook
  
  \begin{align*}
    \varepsilon_i = \set{P_i u + q_i: \norm{u}_2 \leq 1}, i=1,..,K+L, P_i \in S^n
  \end{align*}

  Find a feasible hyperplane strictly separating $\varepsilon_1,..,\varepsilon_K$ from $\varepsilon_{K+1},..,\varepsilon_{K+L}$.
  \begin{align*}
    a^Tx+b > 0, x \in \bigcup_{i=1}^K \varepsilon_i\\
    a^Tx+b < 0, x \in \bigcup_{i=K+1}^{K+L} \varepsilon_i\\
    \text{let } \epsilon > 0 \text{, a constant for strict separation}\\
    \text{relax inequalities to:}\\
    a^Tx+b \leq -\epsilon, x \in \bigcup_{i=1}^K \varepsilon_i\\
    a^Tx+b \geq \epsilon, x \in \bigcup_{i=K+1}^{K+L} \varepsilon_i\\
    a^T(P_iu+q_i)+b \leq -\epsilon, \norm{u}_2 \leq 1, i=[1,K]\\
    a^T(P_iu+q_i)+b \geq \epsilon, \norm{u}_2 \leq 1, i=[K+1,K+L]\\
    \sup_{\norm{u}_2 \leq 1} a^TP_iu + a^Tq_i + b \leq -\epsilon, i=[1,K]\\
    \sup_{\norm{u}_2 \leq 1} a^TP_iu = \frac{a^TP_i (a^TP_i)^T}{\norm{a^TP_i}_2} = \norm{a^TP_i}_2\\
    \norm{a^TP_i}_2 + a^Tq_i + b \leq -\epsilon, i=[1,K]\\
    \norm{a^TP_i}_2 \leq -a^Tq_i - b  -\epsilon, i=[1,K]\\
    \inf_{\norm{u}_2 \leq 1} a^TP_iu + a^Tq_i + b \geq \epsilon, i=[K+1,K+L]\\
    \inf_{\norm{u}_2 \leq 1} a^TP_iu = \frac{a^TP_i (-a^TP_i)^T}{\norm{a^TP_i}_2} = -\norm{a^TP_i}_2\\
    -\norm{a^TP_i}_2 + a^Tq_i + b \geq \epsilon, i=[K+1,K+L]\\
    \norm{a^TP_i}_2 \leq a^Tq_i + b -\epsilon, i=[K+1,K+L]\\
  \end{align*}
  Second order cone formulation:
  \begin{align*}
    &\min_{a,b} 0\\
    &\norm{a^TP_i}_2 \leq -a^Tq_i - b  -\epsilon, i=[1,K]\\
    &\norm{a^TP_i}_2 \leq a^Tq_i + b -\epsilon, i=[K+1,K+L]\\
    &\text{where } \epsilon > 0
  \end{align*}

  \pagebreak
  
\item Q 4.30 textbook
\item Q 4.43(a-b) textbook
  
\end{enumerate}


\end {document}
