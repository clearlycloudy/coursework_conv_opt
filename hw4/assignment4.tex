\documentclass[12pt,letter]{article}

%% \usepackage[fleqn]{amsmath}
\usepackage[margin=1in]{geometry}
\usepackage{amsmath,amsfonts,amsthm,bm}
\usepackage{breqn}
\usepackage{amsmath}
\usepackage{amssymb}
\usepackage{tikz}
\usepackage{algorithm2e}
\usepackage{siunitx}
\usepackage{graphicx}
\usepackage{subcaption}
%% \usepackage{datetime}
\usepackage{multirow}
\usepackage{multicol}
\usepackage{mathrsfs}
\usepackage{fancyhdr}
\usepackage{fancyvrb}
\usepackage{parskip} %turns off paragraph indent
\usepackage{float}
\usepackage{empheq}

\pagestyle{fancy}

\usetikzlibrary{arrows}

\DeclareMathOperator*{\argmin}{argmin}
\newcommand*{\argminl}{\argmin\limits}

\newcommand{\mathleft}{\@fleqntrue\@mathmargin0pt}
\newcommand{\R}{\mathbb{R}}
\newcommand{\Z}{\mathbb{Z}}
\newcommand{\N}{\mathbb{N}}
\newcommand{\ppartial}[2]{\frac{\partial #1}{\partial #2}}
\newcommand{\norm}[1]{\|#1\|}
\newcommand{\set}[1]{\{#1\}}
\newcommand{\notimplies}{\;\not\!\!\!\implies}

\setcounter{MaxMatrixCols}{20}

\begin {document}

  % \begin{cases}
  %     0, & \text{if}\ a=1 \\
  %     1, & \text{otherwise}
  %   \end{cases}

\lhead{Convex Optimization - HW4}
\rhead{(Bill) Yuan Liu, 996954078, 2020/03/20}

\begin{enumerate}
  \item SDP Relaxation and Heuristics for Two-Way Partitioning Problem
  \begin{enumerate}
  \item Q 5.39 textbook\\
    \begin{align*}
      min\ x^T W x\\
      s.t.\ x_i^2 = 1, \forall i \in \{1,..,n\}\\
    \end{align*}
    \begin{enumerate}
    \item
    Show that the two-way partitioning problem can be cast as
    \begin{align*}
      min\ tr(WX)\\
      s.t.\ X \succeq 0, rank(X)=1\\
      X_{ii}=1, \forall i \in \{1,..,n\}
    \end{align*}

    \begin{align*}
      &x^T W x = tr(x^T W x)=tr(Wxx^T)\\
      &let\ X=xx^T\\
      &(\forall i) x_i^2 = 1 \iff x_i = \{-1,1\} \implies x^TIx = n\\
      &x^TIx = tr(xx^T)=n\\
      &((\exists i)X_{ii}=-1 \wedge (\forall i,j) X_{ij} = \{-1,1\} \implies tr(X) < n)\\
      &thus, (\forall i)X_{ii}=1\ for\ tr(X)=n\\
      \\
      &X=xx^T=x
      \begin{bmatrix}
        a_1 & a_2 & .. & a_n
      \end{bmatrix} =
                         \begin{bmatrix}
                           a_1 x & a_2 x & .. & a_n x
                         \end{bmatrix}, a_i \in \R, x \in \R^n\\
      &(\forall i)(\exists j) \beta_{ij} a_i x = a_j x \implies \beta_{ij} a_i x - a_j x = 0\\
      &let\ \gamma_{ij} = \beta_{ij} a_i - a_j\\
      &\gamma_{ij} x = 0\\
      &x\not=0 \implies (\forall i)(\exists j) \gamma_{ij} = 0 \implies linear\ dependence\ between\ column\ vectors\ of\ X\\
      &thus,\ rank(X)=1\\
      \\
      &(\forall w) w^TXw = w^Txx^Tw = (x^Tw)^T x^Tw\\
      &(\forall i)(\forall w)(x^Tw)_i (x^Tw)_i \geq 0 \implies (\forall w)(x^Tw)^T (x^Tw) \geq 0 \iff X\ is\ SPD\\
      \\
      &Combining\ all\ constraints\ and\ objective\ forms\ the\ desired\ result
    \end{align*}
    \pagebreak
  \item
    SDP relaxation of two-way partitioning problem. Using the formulation in part (a), we can form the relaxation:
    \begin{align*}
      min tr(W X)\\
      s.t.\ X \succeq 0\\
      X_{ii} = 1, \forall i \in \{1,..,n\}
    \end{align*}
    This problem is an SDP, and therefore can be solved efficiently. Explain why its optimal value gives a lower bound on the optimal value of the two-way partitioning problem (5.113). What can you say if an optimal point $X^*$ for this SDP has rank one?
    \begin{align*}
      X \succeq 0 \wedge X_{ii} = 1, \forall i \in \{1,..,n\} \implies\ (\forall i) \lambda_i(X) \geq 1\\
      
    \end{align*}
    \end{enumerate}
  \item Q 11.23(b-d) textbook
  \end{enumerate}
  \item Interior Point Method
\end{enumerate}

\end {document}
